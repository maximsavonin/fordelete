\documentclass{article}
\usepackage{amsmath}
\usepackage[utf8]{inputenc}
\usepackage[russian]{babel}

\title{Лабораторная работа}
\author{Савонин Максим}
\date{December 2021}

\begin{document}

\maketitle

\TeX – система компьютерной вёрстки, разработанная американским профессором информатики Дональдом \textit{Кнутом} в конце 70-х годов XX века в целях создания компьютерной типографии. 
\textbf{В отличие от обыкновенных текстовых процессоров и систем компьютерной вёрстки, построенных по принципу WYSIWYG (What You See Is What You Get), в \TeX’е пользователь лишь задает текст и его структуру, а \TeX самостоятельно на основе выбранного пользователем шаблона форматирует документ, заменяя при этом дизайнера и верстальщика.} Документы набираются на собственном языке разметки в виде обычных ASCII-файлов, содержащих информацию о форматировании текста или выводе изображений. Эти файлы (обычно имеющие расширение «.tex») транслируются специальной программой в файлы «.dvi» (device independent — «независимые от устройства»), которые могут быть отображены на экране или напечатаны. DVI-файлы можно специальными программами преобразовать в PostScript, PDF или другой электронный формат.
Ядро \TeX’а представляет собой язык низкоуровневой разметки, содержащий команды отступа и смены шрифта. Огромные возможности в \TeX’е предоставляют готовые наборы макросов и расширений. Наиболее распространённые расширения стандартного \TeX’а (наборы шаблонов, стилей и т. д): LaTeX и AMS-TeX, BibTeX.
LaTeX (произносится   латех)   наиболее популярный набор макрорасширений (или макропакет) системы компьютерной вёрстки \TeX, который облегчает набор сложных документов. Термин LaTeX относится только к языку разметки, он не является текстовым редактором.
\textsl{Пакет LaTeX позволяет автоматизировать многие задачи набора текста и подготовки статей, включая набор текста на нескольких языках, нумерацию разделов и формул, перекрёстные ссылки, размещение иллюстраций и таблиц, ведение библиографии и др.}
Существует несколько наиболее распространённых комплектов вёрстки на основе \TeX’а: \TeX Live и MikTeX (Windows), \TeX Live (UNIX-подобныt системs), MacTeX (Mac OS).

\begin{equation}
\int \limits_S \left( \frac{\partial Q}{\partial x} - \frac{\partial P}{\partial y} \right)\, dx \, dy =\oint \limits_C P\,dx + Q \, dy
\end{equation}

\begin{equation}
\begin{Vmatrix} C_{1}+C & -C \\ -C & C_{1}+C1 \end{Vmatrix}
\end{equation}

\end{document}
